%!TEX platex=1
\documentclass[11pt,xcolor=dvipsnames,table,dvipdfmx]{beamer}
\usepackage{amsmath, amssymb}
\usepackage{latexsym}
\usepackage{ascmac}
\usepackage{bm}

%Beamerの設定
\usetheme{Boadilla}
%Beamerフォント設定
\usepackage{txfonts} % TXフォント
\usepackage[deluxe]{otf} % 日本語多ウェイト化
\renewcommand{\familydefault}{\sfdefault}  % 英文をサンセリフ体に
\renewcommand{\kanjifamilydefault}{\gtdefault}  % 日本語をゴシック体に
\usefonttheme{professionalfonts}
\setbeamerfont{alerted text}{series=\bfseries} % Alertを太字
\setbeamerfont{section in toc}{series=\mdseries} % 目次は太字にしない
\setbeamerfont{frametitle}{size=\Large} % フレームタイトル文字サイズ
\setbeamerfont{title}{size=\LARGE} % タイトル文字サイズ
\setbeamerfont{date}{size=\small}  % 日付文字サイズ

%Beamer色設定
\definecolor{UniBlue}{RGB}{0,150,200} 
\definecolor{AlertOrange}{RGB}{255,76,0}
\definecolor{AlmostBlack}{RGB}{38,38,38}
\setbeamercolor{normal text}{fg=AlmostBlack}  % 本文カラー
\setbeamercolor{structure}{fg=UniBlue} % 見出しカラー
\setbeamercolor{block title}{fg=UniBlue!50!black} % ブロック部分タイトルカラー
\setbeamercolor{alerted text}{fg=AlertOrange} % \alert 文字カラー
\mode<beamer>{
    \definecolor{BackGroundGray}{RGB}{254,254,254}
    \setbeamercolor{background canvas}{bg=BackGroundGray} % スライドモードのみ背景をわずかにグレーにする
}

%フラットデザイン化
\setbeamertemplate{blocks}[rounded] % Blockの影を消す
\useinnertheme{circles} % 箇条書きをシンプルに
\setbeamertemplate{navigation symbols}{} % ナビゲーションシンボルを消す
\setbeamertemplate{footline}[frame number] % フッターはスライド番号のみ

%タイトルページ
\setbeamertemplate{title page}{%
    \vspace{2.5em}
    {\usebeamerfont{title} \usebeamercolor[fg]{title} \inserttitle \par}
    {\usebeamerfont{subtitle}\usebeamercolor[fg]{subtitle}\insertsubtitle \par}
    \vspace{1.5em}
    \begin{flushright}
        \usebeamerfont{author}\insertauthor\par
        \usebeamerfont{institute}\insertinstitute \par
        \vspace{3em}
        \usebeamerfont{date}\insertdate\par
        \usebeamercolor[fg]{titlegraphic}\inserttitlegraphic
    \end{flushright}
}

% graphicx.sty
\usepackage{graphicx}

%
\def\qed{\hfill $\Box$}

\AtBeginSection[]{
    \frame{\tableofcontents[currentsection, hideallsubsections]} %目次スライド
}

%タイトル
\title{common lispで自動微分アルゴリズム}
\author{\textbf{Yuichiro Honda}}
\date{2016/09/16}
\begin{document}
\maketitle

\section{自動微分とは}
\begin{frame}{問題意識}
 計算機上で微分(勾配ベクトル)を速く,正確に計算したい\\
 (性質のいい)関数が与えられたとき,その微分も一意的に定まるはずだが,現状多くの場合微分の関数も手書きで別に与えなければならない \\
 微分の用途:
 \begin{itemize}
  \item 最適化アルゴリズム
  \item 感度分析
  \item 物理モデリング
  \item 機械学習 etc...
 \end{itemize}
 微分は極限を扱うが,計算機には「無限」が理解できない.\\
 → 近似の必要性
\end{frame}

\begin{frame}{定式化}
 入力:関数$f$,点$x$ \\
 出力:入力関数の点$x$での微分値
\end{frame}

\begin{frame}{近似}
 よく見る微分の姿:
 \begin{itemize}
  \item 数値微分:
	\[
	 \frac{df(x)}{dx} = \lim_{h \to 0} \frac{f(x+h)-f(x)}{h}
	\]
  \item 数式微分:
	\begin{align*}
	 \frac{d(u(x)+v(x))}{dx} = \frac{du(x)}{dx}+\frac{dv(x)}{dx} \\
	 \frac{du(x)v(x)}{dx} = u(x)\frac{dv(x)}{dx} + \frac{du(x)}{dx}v(x)
	\end{align*}
 \end{itemize}
\end{frame}

\begin{frame}{近似}
 数値微分の難点:
 \begin{itemize}
  \item 一般的な近似: $\frac{df(x)}{dx} = \frac{f(x+h)-f(x)}{h}$ ($h$は適当に小さい値)
  \item 差分法を用いることによって生じる離散化誤差
 \end{itemize}
 数式微分の難点:
 \begin{itemize}
  \item $\frac{d(make-point(x, y))}{dx}$など,別の場所で定義された関数への操作が難しい.
  \item $\frac{d(u_1(x)u_2(x)...u_{1000}(x))}{dx}$とかだと全部バラす前にスタックオーバーフローしないか心配...
 \end{itemize}
\end{frame}

\begin{frame}{自動微分}
 \begin{block}{自動微分の原理}
  \begin{align*}
   f&(x + \epsilon) = f(x) + f'(x)\epsilon \\
   &\text{$f$: 点$x$で微分可能な任意の関数} \\
   &\text{$\epsilon$: \alert{二重数}の複零元}
  \end{align*}
 \end{block}
\end{frame}

\section{二重数}
\begin{frame}{二重数}
 二重数:複零元(二乗すると$0$になる数)$\epsilon$を用いた実数の拡張 \\
 任意の二重数は$a+b\epsilon$の形で表せる
 \begin{exampleblock}{二重数の基本演算}
  \begin{align*}
   &\text{加算:}(a + b\epsilon) + (c + d\epsilon) = (a + c) + (b + d)\epsilon \\
   &\text{乗算:}(a + b\epsilon)(c + d\epsilon) = ac + (ad + bc)\epsilon \\
   &\text{除算:}\frac{a + b\epsilon}{c + d\epsilon} = \frac{(a + b\epsilon)(c - d\epsilon)}{(c + d\epsilon)(c - d\epsilon)} = \frac{ac + (bc - ad)\epsilon}{c^2}
  \end{align*}
 \end{exampleblock}
\end{frame}

\begin{frame}{二重数}
 \begin{exampleblock}{二重数と関数}
  \begin{align*}
   f(a + b\epsilon) &= \sum_{n = 0}^{\infty}\frac{f^{(n)}(a)(b\epsilon)^n}{n!}\hspace{1.0cm}\text{(テイラー展開)} \\
   &= f(a) + \frac{f'(a)b\epsilon}{1!} + \frac{f''(a)(b\epsilon)^2}{2!} + ... \\
   &= f(a) + bf'(a)\epsilon
  \end{align*}
 \end{exampleblock}
\end{frame}

\begin{frame}{二重数} 
 \begin{exampleblock}{二重数の基本演算(初等関数)}
  \begin{align*}
   &e^{a + b\epsilon} = e^a\sum_{n = 0}^{\infty}\frac{(b\epsilon)^n}{n!} = e^a(1 + b\epsilon) \\
   &\sin(a + b\epsilon) = \sum_{n = 0}^{\infty}\frac{\sin^{(n)} a (b\epsilon)^n}{n!} = \sin a + b\epsilon \cos a \\
   &\cos(a + b\epsilon) = \sum_{n = 0}^{\infty}\frac{\sin^{(n)} a (b\epsilon)^n}{n!} = \cos a - b\epsilon \sin a
  \end{align*}
 \end{exampleblock}
\end{frame}

\section{実装}
\begin{frame}{実装}
 実装の手順
 \begin{enumerate}
  \item 二重数(dual-number)を構造体によって定義
  \item numberに適用できる標準的なメソッドをgenericに再定義し,dual-numberに適用できるようにする
  \item 入力関数$f$,点$a$に対して$f(a + \epsilon)$を計算
  \item 3.で計算した$\epsilon$の係数を取り出す
 \end{enumerate}
\end{frame}


\section{まとめ}
\begin{frame}{できたこと}
  \begin{itemize}
   \item 加減乗除,初等関数($sin x$,$cos x$,$e^x$,$\sqrt{x}$)の組み合わせで表される関数の微分を実現した
   \item lispでpackageの作り方を学習した
   \item 二重数の代数的理解が深まった
  \end{itemize}
\end{frame}

\begin{frame}{できなかったこと}
 \begin{itemize}
  \item 数多あるメソッドの再定義(大小比較,$log$,$mod$とかは入れたかった)
  \item 零除算などの例外処理
  \item マクロをうまくつかう
  \item 高速化
 \end{itemize}
\end{frame}

\begin{frame}{感想}
 common lispには演算子オーバーロードというか演算子が無いのでメソッドをgenericに再定義する手法を取ったが,再定義したメソッドにmapcarなどをかけられないという制約が痛かった.マクロなどをうまく使えば解決できるかもしれないし,そもそももっといい方法があるかもしれない.
\end{frame}

\end{document}
